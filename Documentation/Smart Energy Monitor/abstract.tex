\begin{titlepage}
\thispagestyle{empty}
%\pagestyle{empty}
\begin{center}
\begin{LARGE}
\bf {Abstract}
\end{LARGE}

\end{center}
\large
In 2014, the Electricity Consumption per capita in India was 805.6KWh which is equivalent to 637.43 kg of oil per capita. Over 58 percent of this electricity is produced from non renewable sources of energy. Our dependence on production of energy from non renewable sources of energy makes India both a major greenhouse gas emitter and one of the most vulnerable countries in the world to projected climate change. The country is already experiencing changes in climate and the impacts of climate change, including water stress, heat waves and drought, severe storms and flooding, and associated negative consequences on health and livelihoods. It is imperative that India rapidly adopts renewable sources of energy like solar and wind. But in addition to that it is also the responsibility of the Indian people to monitor their energy consumption and reduce their carbon footprint. 
The Smart Energy Monitor helps the Indian consumer to reduce and monitor their household energy consumption by providing insights to consumption of electricity by individual electrical appliances. The Smart Energy Monitor  connects directly to your electricity panel and uses a mobile app to tell you what devices and appliances are drawing power and when. The monitor listens to the electronic signature of each device and uses algorithms to identify them and monitor their power consumption. It also presents real-time and historical usage for each device. It will help the consumer track energy inefficient appliances and also their monthly usage. From these insights the consumer can reduce their electricity consumption thereby reducing their carbon footprint. 





\end{titlepage}