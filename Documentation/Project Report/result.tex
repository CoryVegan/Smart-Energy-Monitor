\chapter{Results}
\section{Observations}
The sensor is successfully interfacing with the UI using the Raspberry Pi. It is printing the estimated power consumption using output of the current sensors. Using a simple classification algorithm, we are able to determine the device or combination of devices being used in the box. This is meant to portray a room of electronic devices.
The current is being found by sampling the AC output of the sensor and finding its RMS voltage. Upon testing it is found that the output is being found with a 5 percent margin of error.

\section{Drawback }
The classification algorithm being used is only able to identify events with significantly different power consumption. For more precise identification, an algorithm must be implemented that takes other factors such as weather or time into account.
While the margin of error in current reading isn't significant as of now, for a large number of similar devices, 2-5 Watt errors will be substantial. Moreover, only considering current and taking a constant voltage will give inaccurate readings since it doesn't take into account the voltage fluctuations.

\section{New approach}
A KNN algorithm will be able to take into account other factors when identifying events. It can take the power output as well as 2 or e more factors to further aid in identifying which event is most likely to be occurring at the moment.
To improve power reading accuracy, a voltage sensor must also be implemented. Similarly, the sampling code for the current sensor must be refined to reduce error.
Finally, more features need to be added to the UI such as power consumption per device and predictive alerts that determine when a device is being used unnecessarily. There also needs to be a portable version of the interface for accessibility purposes.